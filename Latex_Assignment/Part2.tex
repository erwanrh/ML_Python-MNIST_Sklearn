\section{Part 2}

\subsection{Loss Function}
Our custom loss function is an accuracy score with a penalty on inter-class errors. If the predicted value is different than the true value then the error count will increase +1. We decided to separate the data into two classes : 
\begin{table}[hbt]
\centering
  \begin{tabular}{l|cccccccccc}
   Label  & 0 & 1 & 2 & 3 & 4 & 5 & 6 & 7 & 8 & 9  \\
    \hline
   Class & 0 & 0 & 0 & 0 & 0 & 1 & 1 & 1 & 1 & 1 \\
  \end{tabular}
\end{table}

The counts for each class and label i are as follows : 

\begin{table}[hbt]
\centering
\makegapedcells
\begin{tabular}{cc|cc}
\multicolumn{2}{c}{}
            &   \multicolumn{2}{c}{Predicted} \\
    &       &   True &  False              \\ 
    \cline{2-4}
\multirow{2}{*}{\rotatebox[origin=c]{90}{Actual}}
    & True   & $tp_i$   & $fn_i$                 \\
    & False    & $fp_i$  & $tn_i$                \\ 
    \cline{2-4}
    \end{tabular}

\end{table}

We define the Recall score for one class or label i as : 
\begin{equation*}
	Recall_i = \frac{tp_i}{tp_i+fn_i}
\end{equation*}
which is the well classified individuals on the number of individuals actually in this class. We compute a weighted accuracy on the labels (0:9) and one on the classes (0-1). The formula we use is : 
\begin{equation*}
	Weighted\_Accuracy = {\sum^{K}_{i=1} Recall_i \over K}
\end{equation*}

We join both metrics by multiplying them to have a penalized weighted accuracy of the labels : 
\begin{equation*}
	Score  = Weighted\_Accuracy_{Label} *  Weighted\_Accuracy_{Class}
\end{equation*}



\subsection{Pipeline}
To scales the features, we chose to use a MaxAbsScaler on the data because we want to minimize their variance knowing by constraining them to be between 0 and 1. In fact, our features represent a color code that goes from 0 to 255 which can be easily scaled to a [0,1] interval without any loss of information. We trains a linear SVM model (using the SVC). 

\subsection{Algorithm}
